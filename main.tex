\documentclass[a4paper]{scrreprt}
\usepackage[utf8]{inputenc}
\usepackage[german]{babel}
\usepackage{amsmath}
\usepackage{amsthm}
\usepackage[backend=biber,style=authortitle]{biblatex}
\usepackage[babel]{csquotes}

\newtheorem{definition}{Definition}

\addbibresource{literature.bib}

\title{Untere Schranken für die Abfragekomplexität 
       monotoner symmetrischer Eigenschaften}
\subtitle{Facharbeit}
\author{Erik Diessel}
\begin{document}

\maketitle

\begin{abstract}
Diese Facharbeit beschäftigt sich mit unteren Schranken
für die Abfragekomplexität von monotonen symmetrischen
Eigenschaften von Bitfolgen, besonders für Graph-Eigenschaften.
\end{abstract}

\tableofcontents

\chapter{Die Abfragekomplexität von monotonen Graph-Eigenschaften}
\section{Grundlegende Definitionen}
Im graphentheoretischen Teil folgen wir größtenteils
der Notation eines Standardwerkes der Graphentheorie \footnote{\cite{diestel}}.

\begin{definition}[Graph-Eigenschaft]
Eine Teilmenge $P_n \subseteq \mathcal{G}_n$ der Graphen mit
$n$ Ecken heißt \emph{Graph-Eigenschaft} auf $n$ Ecken,
wenn sie invariant unter Isomorphie ist,
d.h. wenn für zwei isomorphe Graphen $G, G'$ gilt:
$G \in P \iff G' \in P$.
\end{definition}

\begin{definition}[Monotone Graph-Eigenschaft]
Eine Graph-Eigenschaft $P_n$ heißt \emph{monoton}, wenn das
Hinzufügen von Kanten die Eigenschaft erhält,
d.h. wenn ein Graph
$G' \in \mathcal{G}_n$ ein Supergraph von
$G \in \mathcal{G}_n$ ist, so gilt 
$G \in P_n \implies G' \in P_n$.
\end{definition}


\printbibliography
\end{document}
