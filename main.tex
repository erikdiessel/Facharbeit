\documentclass[a4paper]{scrreprt}
\usepackage[utf8]{inputenc}
\usepackage[german]{babel}
\usepackage{amsmath}
\usepackage{amsthm}
\usepackage[backend=biber,style=authortitle]{biblatex}
\usepackage[babel]{csquotes}
\usepackage{tikz}


\newtheorem{definition}{Definition}
\newtheorem{Satz}{Satz}
\theoremstyle{definition}
\newtheorem{example}{Beispiel}

\addbibresource{literature.bib}

\DeclareMathOperator\Min{Min}

\title{Untere Schranken für die Abfragekomplexität 
       monotoner symmetrischer Eigenschaften}
\subtitle{Facharbeit}
\author{Erik Diessel}
\begin{document}

\maketitle

\begin{abstract}
Diese Facharbeit beschäftigt sich mit unteren Schranken
für die Abfragekomplexität von monotonen symmetrischen
Eigenschaften von Bitfolgen, besonders für Graph-Eigenschaften.
\end{abstract}

\tableofcontents

\chapter{Die Abfragekomplexität von monotonen Graph-Eigenschaften}
\section{Grundlegende Definitionen}
Im graphentheoretischen Teil folgen wir größtenteils
der Notation eines Standardwerkes der Graphentheorie \footnote{\cite{diestel}}. \\
Mit $\mathcal{G}_n$ bezeichnen wir die Menge aller
(ungerichteten) Graphen mit $n$ Ecken. \\
Den vollständigen Graphen auf $n$ Ecken bezeichnen wir
mit $K_n$. \\
Ist $G \in \mathcal{G}_n$ ein Subgraph von $G' \in \mathcal{G}_n$,
so schreiben wir $G \subseteq G$ oder $G' \supseteq G$. Ist $G$ ein
echter Subgraph von $G'$ (d.h. $E(G) \neq E(G')$), dann schreiben
wir $G \subset G'$ bzw. $G' \subset G$.


\begin{definition}[Graph-Eigenschaft]
Eine Teilmenge $P_n \subseteq \mathcal{G}_n$ der Graphen mit
$n$ Ecken heißt \emph{Graph-Eigenschaft auf $n$ Ecken},
wenn sie invariant unter Isomorphie ist,
d.h. wenn für zwei isomorphe Graphen $G, G'$ gilt:
$G \in P \iff G' \in P$.
\end{definition}
Statt $G \in P_n$ sagen wir auch: $G$ erfüllt die Eigenschaft $P_n$.


\begin{definition}[Monotone Graph-Eigenschaft]
Eine Graph-Eigenschaft $P_n$ heißt \emph{monoton}, wenn das
Hinzufügen von Kanten die Eigenschaft erhält,
d.h. wenn ein Graph
$G' \in \mathcal{G}_n$ ein Supergraph von
$G \in \mathcal{G}_n$ ist, so gilt 
$G \in P_n \implies G' \in P_n$.
\end{definition}

\begin{example}[Monotone Graph-Eigenschaft]
\label{exmpl:Monotonie}
Wir betrachten die folgende Graph-Eigenschaft 
$P_4 \subseteq \mathcal{G}_4$ auf $4$ Ecken:
$$P_4 := \{ G \in \mathcal{G}_4 \ | \ G \text{ enthält den Subgraphen } K_3 \}$$
Diese Eigenschaft ist \emph{monoton}, da wenn 
$G \in P_4$ gilt, $G$ den Subgraphen $K_3$
enthält, woraus folgt, dass auch jeder Supergraph $G'\supseteq G$
den Subgraphen $K_3$ enthält. 
\end{example}

Im folgenden werden wir uns auf monotone Graph-Eigenschaften beschränken.
Um einige degenerierte Fälle auszuschließen, beschränken
wir uns auch auf \emph{nichttriviale} Eigenschaften.
\begin{definition}[Nichttriviale Graph-Eigenschaft]
Eine monotone Graph-Eigenschaft $P_n$ heißt \emph{nichttrivial},
wenn der leere Graph $E_n$ nicht in $P_n$ enthalten ist und
$P_n\neq \emptyset$ gilt.
\end{definition}
Die Bedingung $E_n \notin P_n$ bedeutet, dass $P_n$ nicht
alle Graphen auf $n$ Ecken beinhaltet. In diesem Fall würde
es keinen Sinn machen, einen Algorithmus berechnen zu lassen,
ob ein gegebener Graph die Eigenschaft erfüllt, da das Ergebnis
unabhängig vom Graphen wäre.

\begin{definition}[Minimaler Graph]
Ein Graph $G \in \mathcal{G}_n$ heißt
\emph{minimaler Graph der montonen Graph-Eigenschaft $P_n$},
wenn $G \in P_n$ gilt, aber jeder echte Subgraph 
$G' \subset G$ die Eigenschaft $P_n$ nicht erfüllt.
\end{definition}

In Beispiel \ref{exmpl:Monotonie} wäre 
\begin{center}
\begin{tikzpicture}[main_node/.style={circle,fill=black,minimum size=0.8em,inner sep=2pt]}]

    \node[main_node] (1) at (0,0) {};
    \node[main_node] (2) at (-0.5, -0.75)  {};
    \node[main_node] (3) at (0.5, -0.75) {};
    \node[main_node] (4) at (1.2, -0.3) {};

    \draw (1) -- (2) -- (3) -- (1);
\end{tikzpicture}
\end{center}

der bis auf Isomorphie eindeutige \emph{minimale} 
Graph von $P_4$, da das Entfernen einer Kante
den Subgraphen $K_3$ zerstören würde.

\begin{definition}[Minimale Graphenmenge $\Min(P_n)$]
Die Menge der minimalen Graphen der monotonen Graph-Eigenschaft
$P_n$ heißt:
$$\Min(P_n) := \{ G\in P_n \ | \ G \text{ ist minimal } \}$$
\end{definition}

Die Menge $\Min(P_n)$ charakterisiert die
Eigenschaft vollständig, wie folgender Satz zeigt:

\begin{Satz}[$\Min(P_n)$ legt $P_n$ eindeutig fest]
Sind $P_n, P'_n \subseteq \mathcal{G}_n$ zwei monotone 
Graph-Eigenschaften, so gilt
$\Min(P_n) = \Min(P'_n)$ genau dann, wenn $P_n = P'_n$.
\end{Satz}
\begin{proof} \hfill
\begin{description}
\item[$\boldsymbol{\Min(P_n) = \Min(P'_n) \implies P_n = P'_n}$]
\hfill \\
Sei $G \in P_n$ beliebig. Wir zeigen dann, dass auch $G \in P'_n$
gilt. Aus Symmetriegründen folgt dann, dass für ein beliebiges 
$G' \in P'_n$ gilt: $G' \in P_n$, woraus die Gleichheit der
Eigenschaften $P_n = P'_n$ folgt. \\
Wir setzen $G_0 := G$. Solange es möglich ist, wählen wir nun
eine Kante $e \in E(G_i)$, sodass für $G_{i+1} := G_i - e$  
weiterhin $G_{i+1} \in P_n$ gilt. \\
Es gibt nun ein $i$, für das es nicht mehr möglich ist,
aus $G_i$ eine Kante zu entfernen und dabei die Eigenschaft
zu erhalten (spätestens, wenn $G_i$ leer ist).
Der Graph $G_i$ ist dann per Konstruktion ein minimaler
Graph, folglich gilt $G_i \in \Min(P_n)$. Aufgrund der
Voraussetzung $\Min(P_n) = \Min(P'_n)$ gilt auch 
$G_i \in \Min(P'_n)$. Da die Eigenschaft $P'_n$ monoton ist,
und $G_i \subseteq G$, gilt dann auch $G_i \in P'_n$,
wie gefordert.
\item[$\boldsymbol{P_n = P'_n \implies \Min(P_n) = \Min(P'_n)}$]
\hfill \\
Wegen $P_n = P'_n$ ist $G \in \mathcal{G}_n$ offensichtlich
genau dann ein minimaler Graph von $P_n$, wenn $G$ ein 
minimaler Graph von $P'_n$ ist. Daraus folgt 
$\Min(P_n) = \Min(P'_n)$.
\end{description}
\end{proof}
Der vorherige Satz ermöglicht uns, eine
monotone Eigenschaft einfacher zu beschreiben.

\printbibliography
\end{document}
